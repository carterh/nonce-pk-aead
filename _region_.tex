\message{ !name(hopper-attack.tex)}
\message{ !name(hopper-attack.tex) !offset(-2) }
\section{[HvAL] \texttt{One-bit} Scheme is not Secure}

Here we consider the \texttt{One-bit} scheme of Hopper et al.\
\cite{HvAL}, and show that it does not meet our XXX-security notion.

\begin{figure}[tph]
\begin{center}
\hfpages{.4}
{
\underline{$\enc^H_K(m;(h,N))$}\\[2pt]
for $i=1$ to~$\ell$\\
\nudge $s \getsr \mathcal{C}_h$\\
\nudge $s' \getsr \mathcal{C}_h$\\
\nudge if $H(K \concat \langle N+i \rangle \concat s)=m$ then $c_i \gets s$\\
\nudge else $c_i \getsr s'$\\
\nudge $h \gets h \concat c_i$\\
$N \gets N + \ell$\\
return $((h,N), c_1 c_2 \cdots c_\ell)$

\medskip
\underline{$\dec^H_K(c_1 c_2 \cdots c_\ell;(h,N))$}\\[2pt]
if $\sum_{i=1}^{\ell} H(K \concat \langle N+i \rangle \concat
c_i) > \ell/2$ then $m \gets 1$\\
else $m \gets 0$\\
$N \gets N+\ell$\\
return $(N,m)$
}
{
\underline{Adversary~$\advA_1^H(c_1 c_2 \cdots c_\ell, K, m)$}\\[2pt]
$t \gets \sum_{i=0}^{\ell} H(K \concat \langle i \rangle \concat c_i)$\\
if $\left| t - \frac{\ell}{2} \right| \geq \delta\frac{\ell}{2}$ 
 then return 1\\
return 0

\medskip
\underline{Adversary~$\advA_2^H(c_1 c_2 \cdots c_\ell, m)$}\\[2pt]
$T\gets 0$\\
for $K \in \mathcal{K}$\\
\nudge $t \gets \sum_{i=1}^{\ell} H(K \concat \langle i \rangle \concat c_i)$\\
\nudge if $\left| t - \frac{\ell}{2} \right| \geq \delta\frac{\ell}{2}$ 
 then $T \gets T+1$\\
if $T > \lceil |\mathcal{K}|/2 \rceil$ then return 1\\
return 0
}
\caption{Hopper et al.\ \texttt{One-bit} Scheme with the PRF
  $F_K(x)=H(K \concat x)$, and the attacks.  [Note: can probably
  extend this to arbitrary PRFs by changing the threshold for~$T$
  appropriately... since if any non-negligible fraction~$\alpha$ of keys result
  in ``non-random'' behavior, you won't have a good PRF to start
  with.]  Adversary~$\advA_1$ is for known-key settings,
  while~$\advA_2$ is for (low-entropy) unknown-key settings.  For
  simplicity, we assume that $N=0$ initially, and that $\ell$ is
  even.  The parameter~$\delta$ is determined in the proof of the
  attack.}
\label{fig:one-bit}
\end{center}
\end{figure}

\begin{theorem}(Informal.) When~$H\colon\bits^* \to \bits$ is a random
  oracle, the \texttt{One-bit} scheme $(\enc^H,\dec^H)$ in
  Figure~\ref{fig:one-bit} is not XXX-secure.  In particular,
  adversary~$\advA_1$ achieves advantage [...] 
\end{theorem}
\begin{proof}(sketch.)
Consider the case that $c_1 c_2 \cdots c_\ell$ resulted from
independent samples from the distribution determined by
$\mathcal{C}_h$, where~$h$ takes on whatever is its initial value (and
can be ignored for the attack).  Let $X_i = H(K \concat \langle i
\rangle \concat c_i)$, so that $\Prob{X_i = 1}=\Prob{X_i=0}=1/2$ for
all $i \in [\ell]$, and $t = \sum_{i=1}^{\ell} X_i$ is a Bernoulli
random varible with $\mathbb{E}[t]=\ell/2$.  In this case, a standard Chernoff
bound gives, for $0 < \delta < 1$
\[
\Prob{\advA_1 \Rightarrow 1 \given b=0} = \Prob{\left| t - \frac{\ell}{2} \right| \geq \delta \frac{\ell}{2}} \leq 2e^{-\frac{\ell\delta^2}{6}}
\]
[Note: should work out a concrete value of $\delta$.  If we want to
allow~$\delta$ outside of the given range, need a different two-sided
Chernoff bound.]

Now consider the case that~$b=1$, with message~$m=0$, so that $c_1 c_2 \cdots c_\ell$ are
the result of~$\enc^H_K(0)$.  Recall that, at the time that $c_i$ was
assigned, it was to one of two independent samples $s,s'$ from the
channel distribution.  In particular, it was assigned the value of~$s$
iff $H(K \concat \langle i \rangle \concat s)=0$.  Thus, if $X_i = H(K
\concat \langle i \rangle \concat c_i)$, we have 
\begin{align*}
\Prob{X_i=0} =& 
\Prob{X_i=0 \cap c_i=s} + \Prob{X_i=0 \cap c_i=s'}\\
=& \Prob{c_i=s} + \Prob{X_i=0 \cap c_i=s'}\\
=& \Prob{H(K \concat \langle i \rangle \concat s)=0} \\
 &+ \Prob{H(K \concat \langle i \rangle \concat s')=0 \cap (s'\neq
   s)}
 + \Prob{H(K \concat \langle i \rangle \concat s')=0 \cap (s' =
   s)} \\
=&
\end{align*}

\end{proof}
\message{ !name(hopper-attack.tex) !offset(-101) }
