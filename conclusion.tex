\section{Conclusion}
\label{sec:conclusion}

Low-entropy secrets such as passwords are likely to persist in computer systems
for many years. Their use in encryption leaves resources vulnerable to offline
attack.  Honey encryption can offer valuable additional protection in such
scenarios. HE yields plausible looking plaintexts under decryption with
invalid keys (passwords), so that offline decryption attempts alone are
insufficient to discover the correct plaintext. 
%HE uses less space compard to existing approaches involving explicit
%encryption of lists of decoys (e.g.,~\cite{Bojinov:2010:KLP:1888881.1888904}).
HE also offers a gracefully degrading hedge against partial disclosure of
high min-entropy keys, and, by simultaneously meeting standard PBE security notions
should keys be high entropy, HE never provides worse security than existing PBE schemes.

We showed applications in which HE security upper bounds are 
equal to an adversary's conditional knowledge of the key
distribution, i.e., they min-entropy of keys. These settings have
message space entropy greater than the entropy of keys, but our framework
can also be used to analyze other settings.

A key challenge for HE---as with all schemes involving decoys---is the
generation of plausible honey messages through good DTE construction. We have
described good DTEs for several natural problems. For the case where plaintexts
consist of passwords, e.g., password vaults, the relationship between
password-cracking and DTE construction mentioned above deserves further
exploration. DTEs offer an intriguing way of potentially
repurposing improvements in cracking technology 
to achieve improvements in encryption security by way of HE.

More generally, for human-generated messages (password vaults, e-mail, etc.),
estimation of message distributions via DTEs is interesting as a natural
language processing problem. Similarly, the reduction of security bounds in HE
to the expected maximum load for balls-and-bins problems offers an interesting
connection with combinatorics. The concrete bounds we present can undoubtedly
be tightened for a variety of cases. Finally, a natural question to pursue is
what kinds of HE bounds can be realized in the standard model via, e.g., $k$-wise
independent hashing. 
%Such connections inside and outside cryptography are just
%a few of the many stimulating directions for future research presented by honey
%encryption. 


