\newcommand{\bnm}{\begin{newmath}}
\newcommand{\enm}{\end{newmath}}

\newcommand{\bea}{\begin{eqnarray*}}
\newcommand{\eea}{\end{eqnarray*}}



\newcommand{\bne}{\begin{newequation}}
\newcommand{\ene}{\end{newequation}}

\newenvironment{newmath}{\begin{displaymath}%
\setlength{\abovedisplayskip}{4pt}%
\setlength{\belowdisplayskip}{4pt}%
\setlength{\abovedisplayshortskip}{6pt}%
\setlength{\belowdisplayshortskip}{6pt} }{\end{displaymath}}

\newenvironment{neweqnarrays}{\begin{eqnarray*}%
\setlength{\abovedisplayskip}{-4pt}%
\setlength{\belowdisplayskip}{-4pt}%
\setlength{\abovedisplayshortskip}{-4pt}%
\setlength{\belowdisplayshortskip}{-4pt}%
\setlength{\jot}{-0.4in} }{\end{eqnarray*}}

\newenvironment{newequation}{\begin{equation}%
\setlength{\abovedisplayskip}{4pt}%
\setlength{\belowdisplayskip}{4pt}%
\setlength{\abovedisplayshortskip}{6pt}%
\setlength{\belowdisplayshortskip}{6pt} }{\end{equation}}


\newcounter{ctr}
\newcounter{savectr}
\newcounter{ectr}

\newenvironment{newitemize}{%
\begin{list}{\mbox{}\hspace{5pt}$\bullet$\hfill}{\labelwidth=15pt%
\labelsep=5pt \leftmargin=20pt \topsep=3pt%
\setlength{\listparindent}{\saveparindent}%
\setlength{\parsep}{\saveparskip}%
\setlength{\itemsep}{3pt} }}{\end{list}}


\newenvironment{newenum}{%
\begin{list}{{\rm (\arabic{ctr})}\hfill}{\usecounter{ctr} \labelwidth=17pt%
\labelsep=5pt \leftmargin=22pt \topsep=3pt%
\setlength{\listparindent}{\saveparindent}%
\setlength{\parsep}{\saveparskip}%
\setlength{\itemsep}{2pt} }}{\end{list}}

\newlength{\saveparindent}
\setlength{\saveparindent}{\parindent}
\newlength{\saveparskip}
\setlength{\saveparskip}{\parskip}


\newcommand{\Adv}{\mathbf{Adv}}
\newcommand{\AdvMI}[1]{\Adv^\mathrm{mi}_{#1}}
\newcommand{\AdvKKIND}[1]{\Adv^\mathrm{kk\textnormal{-}rod}_{#1}}
\newcommand{\AdvHEDIST}[1]{\Adv^\mathrm{dist}_{#1}}
\newcommand{\AdvROD}[1]{\Adv^\mathrm{rod}_{#1}}
\newcommand{\AdvleROD}[1]{\Adv^\mathrm{le\textnormal{-}rod}_{#1}}
\newcommand{\AdvMR}[1]{\Adv^\mathrm{mr}_{#1}}
\newcommand{\AdvMRCCA}[1]{\Adv^\mathrm{mr\textnormal{-}cca}_{#1}}
\newcommand{\AdvKR}[1]{\Adv^\mathrm{kr}_{#1}}
\newcommand{\AdvSAMPRAT}[1]{\Adv^\mathrm{dte\textnormal{-}ratio}_{#1}}
\newcommand{\AdvSAMPIND}[1]{\Adv^\mathrm{dte}_{#1}}
%\newcommand{\AdvDTE}[1]{\Adv^\mathrm{dte}_{#1}}

\newcommand{\encOracle}{\textbf{Enc}}
\newcommand{\decOracle}{\textbf{Dec}}

\newcommand{\negsmidge}{{\hspace{-0.1ex}}}
\newcommand{\cdotsm}{\negsmidge\negsmidge\negsmidge\cdot\negsmidge\negsmidge\negsmidge}

\def\suchthatt{\: :\:}

\newcommand{\Prob}[1]{{\Pr\left[\,{#1}\,\right]}}
\newcommand{\probb}[2]{{\Pr}_{#1}\left[\,{#2}\,\right]}
\newcommand{\Probb}[2]{\Pr[#1]}
\newcommand{\CondProb}[2]{{\Pr}\left[\: #1\:\left|\right.\:#2\:\right]}
\newcommand{\CondProbb}[2]{\Pr[#1|#2]}
\newcommand{\ProbExp}[2]{{\Pr}\left[\: #1\:\suchthatt\:#2\:\right]}
\newcommand{\Ex}[1]{{\textnormal{E}\left[\,{#1}\,\right]}}
\newcommand{\Exx}{{\textnormal{E}}}

\newcommand{\true}{\textsf{true}}
\newcommand{\false}{\textsf{false}}



\newcommand{\secref}[1]{\mbox{Section~\ref{#1}}}
\newcommand{\apref}[1]{\ifnum\camready=0 \mbox{Appendix~\ref{#1}}\else the
full version\fi}
\newcommand{\thref}[1]{\mbox{Theorem~\ref{#1}}}
\newcommand{\defref}[1]{\mbox{Definition~\ref{#1}}}
\newcommand{\corref}[1]{\mbox{Corollary~\ref{#1}}}
\newcommand{\lemref}[1]{\mbox{Lemma~\ref{#1}}}
\newcommand{\clref}[1]{\mbox{Claim~\ref{#1}}}
\newcommand{\propref}[1]{\mbox{Proposition~\ref{#1}}}
\newcommand{\factref}[1]{\mbox{Fact~\ref{#1}}}
\newcommand{\remref}[1]{\mbox{Remark~\ref{#1}}}
\newcommand{\figref}[1]{\mbox{Figure~\ref{#1}}}
%\newcommand{\eqref}[1]{\mbox{Equation~(\ref{#1})}}
% Have to use \renewcommand because exists already in amsmath
\renewcommand{\eqref}[1]{\mbox{(\ref{#1})}}
\newcommand{\consref}[1]{\mbox{Construction~\ref{#1}}}
\newcommand{\tabref}[1]{\mbox{Table~\ref{#1}}}


\newcommand{\getsr}{{\:{\leftarrow{\hspace*{-3pt}\raisebox{.75pt}{$\scriptscriptstyle\$$}}}\:}}
\newcommand{\getc}{{\:\leftarrow_{\cdist}\:}}
\newcommand{\getm}{{\:\leftarrow_{\mdist}\:}}
\newcommand{\getd}{{\:\leftarrow_{\ddist}\:}}
%\newcommand{\getm}{{\:{\leftarrow{\hspace*{-3pt}\raisebox{.75pt}{$\scriptscriptstyle \mdist$}}}\:}}
\newcommand{\getk}{{\:\leftarrow_{\kdist}\:}}
%\newcommand{\getk}{{\:{\leftarrow{\hspace*{-3pt}\raisebox{.75pt}{$\scriptscriptstyle \kdist$}}}\:}}
\newcommand{\getx}{{\:\leftarrow_{\xdist}\:}}
\newcommand{\gety}{{\:\leftarrow_{\ydist}\:}}



\newcommand{\gamesfontsize}{\small}
\newcommand{\fpage}[2]{\framebox{\begin{minipage}[t]{#1\textwidth}\setstretch{1.1}\gamesfontsize  #2 \end{minipage}}}

\newcommand{\hpages}[3]{\begin{tabular}{cc}\begin{minipage}[t]{#1\textwidth} #2 \end{minipage} & \begin{minipage}[t]{#1\textwidth} #3 \end{minipage}\end{tabular}}


\newcommand{\hfpages}[3]{\hfpagess{#1}{#1}{#2}{#3}}
\newcommand{\hfpagess}[4]{
        \begin{tabular}{c@{\hspace*{.5em}}c}
        \framebox{\begin{minipage}[t]{#1\textwidth}\setstretch{1.15}\gamesfontsize #3 \end{minipage}}
        &
        \framebox{\begin{minipage}[t]{#2\textwidth}\setstretch{1.15}\gamesfontsize #4 \end{minipage}}
        \end{tabular}
    }
\newcommand{\hfpagesss}[6]{
        \begin{tabular}{c@{\hspace*{.5em}}c@{\hspace*{.5em}}c}
        \framebox{\begin{minipage}[t]{#1\textwidth}\setstretch{1.1}\gamesfontsize #4 \end{minipage}}
        &
        \framebox{\begin{minipage}[t]{#2\textwidth}\setstretch{1.1}\gamesfontsize #5 \end{minipage}}
        &
        \framebox{\begin{minipage}[t]{#3\textwidth}\setstretch{1.1}\gamesfontsize #6 \end{minipage}}
        \end{tabular}
    }
\newcommand{\hfpagessss}[8]{
        \begin{tabular}{c@{\hspace*{.5em}}c@{\hspace*{.5em}}c@{\hspace*{.5em}}c}
        \framebox{\begin{minipage}[t]{#1\textwidth}\setstretch{1.1}\gamesfontsize #5 \end{minipage}}
        &
        \framebox{\begin{minipage}[t]{#2\textwidth}\setstretch{1.1}\gamesfontsize #6 \end{minipage}}
        &
        \framebox{\begin{minipage}[t]{#3\textwidth}\setstretch{1.1}\gamesfontsize #7 \end{minipage}}
        &
        \framebox{\begin{minipage}[t]{#4\textwidth}\setstretch{1.1}\gamesfontsize #8 \end{minipage}}
        \end{tabular}
    }

\newcommand{\vecw}{\mathbf{w}}
\newcommand{\R}{\mathbb{R}}
\newcommand{\N}{\mathbb{N}}
\newcommand{\Z}{\mathbb{Z}}
\newcommand{\load}{L}
\newcommand{\coll}{\mathsf{Coll}}
\newcommand{\nocoll}{\overline{\mathsf{Coll}}}


\newcommand{\Img}{\textsf{Img}}

\newcommand{\queriedM}{\texttt{M}}
\newcommand{\queriedC}{\texttt{C}}
\def \mspace {{\cal{M}}}
\def \mspacebot {{\cal{M}_\bot}}
\def \sspace {{\cal{S}}}
\def \slen {{s}}
\def \kspace {{\cal{K}}}
\def \kspacesize {{m}}
\def \mspacesize {{n}}
\def \kdict {D}
\def \dictsize {d}
\newcommand{\kdist}{p_k}
\newcommand{\dist}{p}
\newcommand{\mdist}{p_m}
\newcommand{\xdist}{p_x}
\newcommand{\ydist}{p_y}
\newcommand{\ddist}{p_d}
\newcommand{\cdist}{p_c}
\newcommand{\cspace}{{\mathcal C}}
%\def \kdist {{\kappa}}
%\def \mdist {{\mu}}
%\def \ddist {{\delta}}
\def \pspace {{\cal{P}}}
\def \mpspace {{\cal{MP}}}
\def \cspace {{\cal{C}}}
\def \key {K}
\def \msg {M}
\def \msgvec {{\vec M}}
\def \seed {S}
\def \ctxt {C}
\def \ctxtvec {{\vec C}}
\def \ctxtpart {C_2}
\def \DTE {{\textsf{DTE}}}
\def \DME {{\textsf{DME}}}
\def \mask {{\nu}}
\newcommand{\genprime}{{\textsf{GenPrime}}}
\newcommand{\isprime}{{\textsf{IsPrime}}}
\newcommand{\divisible}{{\textsf{IsDiv}}}
\newcommand{\LeastLesserPrime}{{\textsf{PrevPrime}}}
\newcommand{\GetPrevDiv}{{\textsf{PrevPrimeDiv}}}
\def \encode {{\textsf{encode}}}
\def \decode {{\textsf{decode}}}
\newcommand{\DTEis}{{\textsf{IS-DTE}}}
\newcommand{\encodeis}{{\textsf{is-encode}}}
\newcommand{\decodeis}{{\textsf{is-decode}}}
\newcommand{\DTErej}{{\textsf{REJ-DTE}}}
\newcommand{\encoderej}{{\textsf{rej-encode}}}
\newcommand{\decoderej}{{\textsf{rej-decode}}}
\def \prng {{\textsf{prng}}}
\def \primetest {{\textsf{primetest}}}
\def \coins {{\xi}}

\newcommand{\eqnand}{\hspace*{2em}\textnormal{and}\hspace*{2em}}



\newcommand{\oddnums}{\mathbb{O}}


\newcommand{\DTErsarej}{{\textsf{RSA-REJ-DTE}}}
\newcommand{\encodeRSAREJ}{{\textsf{rsa-rej-encode}}}
\newcommand{\decodeRSAREJ}{{\textsf{rsa-rej-decode}}}
\newcommand{\DTErsainc}{{\textsf{RSA-INC-DTE}}}
\newcommand{\encodeRSAINC}{{\textsf{rsa-inc-encode}}}
\newcommand{\decodeRSAINC}{{\textsf{rsa-inc-decode}}}
\newcommand{\DTEunf}{{\textsf{UNF-DTE}}}
\newcommand{\DTEnunf}{{\textsf{NUNF-DTE}}}
\newcommand{\DTErsassl}{{\textsf{RSA-SSL-DTE}}}
\newcommand{\encodeRSASSL}{{\textsf{rsa-ssl-encode}}}
\newcommand{\decodeRSASSL}{{\textsf{rsa-ssl-decode}}}



%\newcommand{\encodeis}{{\textsf{encode}_{\textrm{is}}}}
%\newcommand{\decodeis}{{\textsf{decode}_{\textrm{is}}}}
\newcommand{\rep}{\textsf{rep}}
\newcommand{\isErr}{\epsilon_{\textnormal{is}}}
\newcommand{\incErr}{\epsilon_{\textnormal{inc}}}
\def \kg{{\textsf{kg}}}
\def \decode {{\textsf{decode}}}
\def \enc {{\textsf{enc}}}
\def \dec {{\textsf{dec}}}
\def \DMEscheme {{\textsf{DME}}}
\def \Enc {{\textsf{Enc}}}
\def \Dec {{\textsf{Dec}}}

\def \SEscheme {{\textsf{SE}}}
\def \HEscheme {{\textsf{HE}}}
\def \CTR {{\textsf{CTR}}}
\def \encHE {{\textsf{HEnc}}}
\def \HIDE {{\textsf{HiaL}}}
\def \encHIDE {{\textsf{HEnc}}}
\def \decHIDE {{\textsf{HDec}}}
\def \decHE {{\textsf{HDec}}}
\def \encHEt {{\textsf{HEnc2}}}
\def \decHEt {{\textsf{HDec2}}}
%\def \dist {{\textsf{dist}}}
\def \ind {{\textsf{index}}}
\def \salt {{\textsf{sa}}}

\newcommand{\myind}{\hspace*{1em}}
\newcommand{\thh}{^{\textit{th}}} % th
\newcommand{\concat}{\,\|\,}
\newcommand{\dotdot}{..}
\newcommand{\emptystr}{\varepsilon}


\newcommand{\round}{\textsf{round}}

\newcommand{\alphabar}{\overline{\alpha}}
\newcommand{\numbinsbar}{\overline{b}}
\newcommand{\numballs}{a}
\newcommand{\numbins}{b}

%\def \encHE {{\sf{enc}^{HE}}}
%\def \decHE {{\sf{dec}^{HE}}}
%\def \encHEt {{\sf{enc}^{HE2}}}
%\def \decHEt {{\sf{dec}^{HE2}}}
\def \idealHE {{\mathcal{HE}}}
\def \IEnc {{\mathbf{\rho}}}
\def \IDec {{\mathbf{\rho^{-1}}}}
\def \OEnc {{\mathbf{Enc}}}
\def \ODec {{\mathbf{Dec}}}
\newcommand{\SimuProc}{\mathbf{Sim}}
\newcommand{\ROProc}{\mathbf{RO}}
\newcommand{\PrimProc}{\mathbf{Prim}}
\def \stm {g}
\def \istm {\hat{g}}
\def \kts {{f}}
\def \lex {{\sf lex}}
\def \part {part}
\def \kd {{\sf{kd}}}
\def \msgdist {{d}}
\def \keydist {{r}}
\def \ind {{\sf{index}}}
\def \kprf {z}
\def \adv {{\cal A}}
\def \pwds {u}
\def \tokens {v}
\def \template{{\cal T}}
\def \vaultset{{\cal V}}
\def \ext {{\sf ext}}
\def \offset {\delta}
\def \maxweight {\epsilon}
\def \advo {{\cal A}^{*}}

\newcommand{\Chall}{\textsf{Ch}}
\newcommand{\MI}{\textnormal{MI}}
\newcommand{\MR}{\textnormal{MR}}
\newcommand{\IND}{\textnormal{IND}}
\newcommand{\KKIND}{\textnormal{KK-ROD}}
\newcommand{\ROD}{\textnormal{ROD}}
\newcommand{\leROD}{\textnormal{le-ROD}}
\newcommand{\MRCCA}{\textnormal{MR-CCA}}
\newcommand{\SAMP}{\textnormal{SAMP}}
\newcommand{\DTEgame}{\textnormal{SAMP}}
\newcommand{\KR}{\textnormal{KR}}
\newcommand{\advA}{{\cal A}}
\newcommand{\advB}{{\cal B}}
\newcommand{\advI}{{\cal I}}
\newcommand{\next}{\;;\;}
\newcommand{\TabC}{\texttt{C}}
\newcommand{\TabR}{\texttt{R}}
\newcommand{\Hash}{H}
\newcommand{\Cipher}{\pi}
\newcommand{\CipherInv}{\pi^{-1}}
\newcommand{\simu}{{\mathcal S}}
\newcommand{\prim}{P}
\newcommand{\maxguess}{\gamma}

\newcommand{\bigO}{\mathcal{O}}
\newcommand{\calG}{{\mathcal G}}

\def\sqed{{\hspace{5pt}\rule[-1pt]{3pt}{9pt}}}
\def\qedsym{\hspace{2pt}\rule[-1pt]{3pt}{9pt}}

\newcommand{\Colon}{{\::\;\;}}
\newcommand{\good}{\textsf{Good}}

\newcommand\Tvsp{\rule{0pt}{2.6ex}}
\newcommand\Bvsp{\rule[-1.2ex]{0pt}{0pt}}
\newcommand{\TabPad}{\hspace*{5pt}}
\newcommand\TabSep{@{\hspace{5pt}}|@{\hspace{5pt}}}
\newcommand\TabSepLeft{|@{\hspace{5pt}}}
\newcommand\TabSepRight{@{\hspace{5pt}}|}




\DeclareMathOperator*{\argmin}{argmin}
\newcommand{\comma}{\textnormal{,}}

\renewcommand{\paragraph}[1]{\vspace*{6pt}\noindent\textbf{#1}\;}


\newcommand{\supp}{\textnormal{Supp}}



%%%%%%%%%%%%%%%%%%%%%%%%%%%%%%%%%%%%%%%%%%%%%%%%%%%%%%%%%%%%%%%%%%%%%%%%%%%%%%
%
% Figure and table macros
%

\newcounter{mytable}

\def\mytable{\begin{centering}\refstepcounter{mytable}}
\def\endmytable{\end{centering}}

\def\mytablecaption#1{\vspace{2mm}
                      \centerline{Table \arabic{mytable}.~{#1}}
                      \vspace{6mm}
             \addcontentsline{lot}{table}{\protect\numberline{\arabic{mytable}}~{#1}}}


\newcounter{myfig}
\def\myfig{\begin{centering}\refstepcounter{myfig}}
\def\endmyfig{\end{centering}}

\def\myfigcaption#1{
             \vspace{2mm}
             \centerline{\textsf{Figure \arabic{myfig}.~{#1}}}
             \vspace{6mm}
             \addcontentsline{lof}{figure}{\protect\numberline{\arabic{myfig}}~{#1}}}


%%%%%%%%%%%%%%%%%%%%%%%%%%%%%%%%%%%%%%%%%%%%%%%%%%%%%%%%%%%%%%%%%%%%%%%%%%%%%%
%
% New commands:
%
\newcommand{\reminder}[1]{ [[[ \marginpar{\mbox{$<==$}} #1 ]]] }

%
% New theorem types:
%
\ifnum\camready=0
\newtheorem{observation}{Observation}
\newtheorem{definition}{Definition}
\newtheorem{claim}{Claim}
\newtheorem{assumption}{Assumption}
\newtheorem{fact}{Fact}
\newtheorem{theorem}{Theorem}
\newtheorem{lemma}{Lemma}
\newtheorem{corollary}{Corollary}
\newtheorem{proposition}{Proposition}
\newtheorem{example}{Example}
\fi

%
% Definitions:
%
\def \blackslug{\hbox{\hskip 1pt \vrule width 4pt height 8pt
    depth 1.5pt \hskip 1pt}}
\def \qed{\quad\blackslug\lower 8.5pt\null\par}
% In-line QED, for ending a proof with a $$ formula
% In-line QED, for ending a proof with a $$ formula
\def \inQED{\quad\quad\blackslug}
\def \Qed{\QED}
\def \QUAD{$\Box$}
\def \Proof{\par\noindent{\bf Proof:~}}
\def \proof{\Proof}
\def \poly {\mbox{$\mathsf{poly}$}}
\def \binary {\mbox{$\mathsf{binary}$}}
\def \ones {\mbox{$\mathsf{ones}$}}
\def \rank {\mbox{$\mathsf{rank}$}}
%\def \bits {\mbox{$\mathsf{bits}$}}
\def \bits {\{0,1\}}
\def \factorial {\mbox{$\mathsf{factorial}$}}
\def \fr {\mbox{$\mathsf{fr}$}}
\def \pr {\mbox{$\mathsf{pr}$}}
\def \zon {\{0,1\}^n}
\def \zo  {\{0,1\}}
\def \zok {\{0,1\}^k}
\def \mo {s}

\newcommand{\Hdot}{H(\mbox{ } \cdot \mbox{ }  , \mbox{ } \del)}

\newcommand{\mynote}[2]{\textcolor{red}{Note from #1: #2}}
\newcommand{\noteari}[1]{\mynote{Ari}{#1}}

\newcounter{mynote}[section]
\newcommand{\notecolor}{blue}
\newcommand{\mythenote}{\thesection.\arabic{mynote}}
\newcommand{\tnote}[1]{\ifnum\authnotes=1\refstepcounter{mynote}{\bf \textcolor{\notecolor}{$\ll$TomR~\mythenote: {\sf #1}$\gg$}}\fi}
\newcommand{\tsnote}[1]{\ifnum\authnotes=1\refstepcounter{mynote}{\bf
    \textcolor{cyan}{$\ll$TomS~\mythenote: {\sf #1}$\gg$}}\fi}
\newcommand{\fixme}[1]{\ifnum\authnotes=1{\textcolor{red}{[[FIXME: #1]]}}\fi}



%Variable names:
%%%%%%%%%%
% TomR: I moved a bunch of macros to tomsdefs.tex so they are all in one place
%%%%%%%%%%

\newcommand\ignore[1]{}

%Notation:

\newcommand\simplescheme{simple}

\newcommand{\calX}{\mathcal{X}}
\newcommand{\calY}{\mathcal{Y}}

%Parameter names:

\newcommand\keylen{\ensuremath{{\sf l}}}
\newcommand\driftspacesize{\ensuremath{{\sf d}}}



\newcommand{\nudge}{\hspace*{2ex}}  
\newcommand{\walk}{\mathcal{W}}
\newcommand{\matP}{\mathbf{P}}
\newcommand{\vecpi}{\boldsymbol{\pi}}
\newcommand{\calS}{\mathcal{S}}
\newcommand{\given}{\,|\,} 
\renewcommand{\bits}{\{0,1\}}
\newcommand{\propose}{\mathsf{propose}}
\newcommand{\decide}{\mathsf{decide}}
\newcommand{\sample}{\mathsf{sample}}
\newcommand{\terminate}{\mathsf{stop}}
\newcommand{\st}{\mathrm{st}}
\newcommand{\params}{\mathrm{params}}
\newcommand{\mhenc}{\mathsf{MH}\enc}
\newcommand{\mhdec}{\mathsf{MH}\dec}
\newcommand{\calP}{\mathcal{P}}
\newcommand{\calQ}{\mathcal{Q}}
\newcommand{\smidge}{\hspace*{1ex}}
\newcommand{\nnudge}{\nudge\smidge}
\newcommand{\xor}{\oplus}

\newcommand{\Kgen}{\mathsf{Key}}
\newcommand{\pubkeys}{\mathcal{K}_P}
\newcommand{\seckeys}{\mathcal{K}_S}
\newcommand{\header}{A}
\newcommand{\adata}{\mathcal{H}}
\newcommand{\pubiv}{N}
\newcommand{\pubivs}{\mathcal{V}_P}
\newcommand{\seciv}{S}
\newcommand{\secivs}{\mathcal{V}_S}
\newcommand{\ptxts}{\mathcal{M}}
\newcommand{\ctxts}{\mathcal{C}}
\newcommand{\IV}{\mathrm{IV}}
\newcommand{\pk}{\mathrm{pk}}
\newcommand{\sk}{\mathrm{sk}}
\newcommand{\encprim}[3]{\enc_{#1}^{#2}(#3)}
\newcommand{\decprim}[3]{\dec_{#1}^{#2}(#3)}
\newcommand{\encprimO}[4]{\enc_{#1}^{#4,(#2)}(#3)}
\newcommand{\decprimO}[4]{\dec_{#1}^{#4,(#2)}(#3)}
\newcommand{\Encprim}[3]{\Enc_{#1}^{#2}(#3)}
\newcommand{\Decprim}[3]{\Dec_{#1}^{#2}(#3)}
\newcommand{\EncprimO}[4]{\Enc_{#1}^{#4,(#2)}(#3)}
\newcommand{\DecprimO}[4]{\Dec_{#1}^{#4,(#2)}(#3)}

\newcommand{\encap}{\mathsf{encap}}
\newcommand{\decap}{\mathsf{decap}}

\newcommand{\ro}{\mathsf{RO}}