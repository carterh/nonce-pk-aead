%%%%%%%%%%%%%%%%%%%%%%%%%%%%%%%%%%%%%%%%%%%%%%%%%%%%%%%%%%%%%%%%%%%%%%%%%%%%%%%%
\section{DTE-Encrypt-DTE Constructions}
\label{sec:gencons}


We investigate a natural methodology for building DME schemes, which can be seen
as generalization of the Rank-Encipher-Unrank approach introduced for
FPE~\cite{BRRS09}. We start by providing a more general treatment of 
distribution-transforming encoders (DTE), originally introduced by Juels and
Ristenpart~\cite{JR14} for use in building HE schemes. 

\paragraph{DTE schemes.} 
A DTE scheme $\DTE = (\encode,\decode,\xdist,\ydist)$ is a pair of algorithms
and pair of distributions.  The possibly
randomized encode algorithm $\encode$ takes as input a value $X \in \calX \equiv
\supp(\xdist)$ and outputs a value $Y \in \calY \equiv \supp(\ydist)$. 
The possibly randomized decode algorithm $\decode$ takes as input a value $Y\in\calY$ and
outputs a value $X\in\calX$. We require that for all $X \in \calX$  
\bnm
  \Prob{\decode(\encode(X)) = X} = 1 
\enm
where the probability is over the random coins used by the two algorithms.
\tnote{We may want to generalize this notion of correctness to match the DME
ones later.} 

In their use by~\cite{JR14} for HE, a DTE was used with  
some distribution $\xdist$ over plaintext messages and $\ydist$ was the uniform distribution over
bit strings of an appropriate length. Here we will also be interested in
DTEs that convert from uniform distributions over bit strings to values that should
appear to be non-uniformly distributed. 
%The goal was to transform values drawn according to $\mdist$ to uniform
%bit strings. 

\paragraph{The construction.} Let $\mdist$ be a message distribution, $\cdist$
be a ciphertext distribution, and $\xdist$ and $\ydist$ be the uniform distribution
over bit strings of lengths that will be defined appropriately below.  Let $\DTE_m =
(\encode_m,\decode_m,\mdist,\xdist)$  and $\DTE_c =
(\encode_c,\decode_c,\ydist,\cdist)$ be DTE schemes.
Let $\SEscheme = (\enc,\dec)$ be a conventional symmetric encryption scheme. 
Then the DTE-Encrypt-DTE scheme $\DME = (\Enc,\Dec)$ has encryption and
decryption defined by 
\bnm
  \Enc(K,M) = \encode_c(\enc(K,\encode_m(M))) \eqnand
  \Dec(K,C) = \decode_m(\dec(K,\decode_c(C))) \;.
\enm
If we set $\mdist$ and $\cdist$ to be uniform, then building a DTE-Encrypt-DTE
scheme is equivalent to building a Rank-Encrypt-Unrank style
scheme~\cite{BRRS09,LDJRS14}. 
\tsnote{If I take Rank as $\encode_m$ and Unrank as
  $\encode_c$, so that I follow your DTE-Encrypt-DTE construction,
  then I don't see what can be asserted about $\mdist$ (although 
$\supp(\mdist)$ is strings of a particular format).
  However, $\ydist$ should be uniform for good Encrypt stages, so
  likewise $\cdist$ uniform over the support of $\cdist$.  Note: even
  if $\supp(\mdist)=\supp(\cdist)$, i.e.\ FPE, we don't want to demand
  $\mdist=\cdist$ if we want ``typical'' kinds of security.}
When $\cdist$ is uniform over bit strings, and
hence $\DTE_c$ can be the identity,  then the resulting DTE-Encrypt scheme 
is equivalent to an HE scheme.
\tsnote{Isn't this under the assumption that $\dec_{K'}(Y)$ is uniform
  over bitstrings when $K'$ is the wrong key and $Y$
  is uniform? That's probably true for natural schemes, even when
  $K'=K$, although not
  for CBC (or other schemes) with padding/redundancy.} 
When $\mdist$ is arbitrary  \tnote{need to define
this} then Encrypt-DTE gives rise to a stegonagraphy scheme with covertext
distribution~$\cdist$. 
\tsnote{A couple of observations: (1) $\xdist$ need not be uniform if
  one uses typical encryption schemes, which are provably secure
  against adversarially chosen $\xdist$; (2) if we want to target
  ``typical'' kinds of security notions for this DME construction,
  we'll need $\mdist$ to be adversarially chosen; this fights against
  (3) if we want HE-type properties, then
  $\xdist$ and $\mdist$ are pretty restricted, based on current
  knowledge; (4) for typical $\enc$, we'll have $\ydist$ uniform-ish.}

In the above we have glossed over the lengths of strings, implicitly assuming
that the domains and ranges of $\DTE_m$, $\DTE_c$, and $\SEscheme$ align
appropriately. We will see how this plays out in detail for particular
constructions below.

The main challenge towards using the DTE-Encrypt-DTE construction is that of
building DTEs that transform uniform bit strings to other distributions. The DTE
schemes explored thus far, including all those in \cite{JR14}, work in the other
direction. One approach for doing so is to leverage the large literature on
random variable sampling algorithms that use uniform random variables to
generate, for example, normal or exponentially distributed random variables.
Unfortunately,  these algorithms do not easily provide DTEs since they are not
invertible (being typically many-to-one). 









