\section{Encryption Schemes}
\label{sec:prelims}
\label{sec:encryption}
We begin by ... [blah blah]

\paragraph{Notational conventions. }When $X,Y$ are bitstrings, we write $X \concat Y$ for their concatenation, $|X|$ for the bitlength, and $X\xor Y$ for their bitwise exclusive-or.  When $\calX$ is a set, we write $x \getsr \calX$ to mean that a value is sampled from $\calX$ and assigned to the variable~$x$.  Unless otherwise specified, sampling from sets is via the uniform distribution.

When~$F$ is an randomized algorithm, we write $x \getsr F(y_1,y_2,\cdots)$ to mean that~$F$ is run with the specified inputs and the result assigned to~$x$.  When~$F$ is determinsitic, we drop the $\$$-embellishment from the assignment arrow.  Algorithms may be provided access to one or more \emph{oracles}, and we write these as superscripts, e.g. $F^{\mathcal{O}}$; oracle access is black box and via an specified interface.  

An \emph{adversary} is a randomized algorithm.

\paragraph{Asymmetric (Public-key) AEAD Schemes. }
Fix sets $\pubkeys, \seckeys, \adata, \pubivs, \secivs, \ptxts,
\ctxts$, the first two of which are nonempty.  An public-key AEAD
(PK-AEAD) scheme $\pkaead=(\Kgen,\Enc,\Dec)$ is a triple of algorithms.  The randomized \emph{key generation} algorithm~$\Kgen$ takes no input and returns a public-key, secret-key pair $(\pk,\sk)$.  We write $(\pk,\sk)\getsr\Kgen$ for the operation of key generation.

The \emph{encryption} algorithm $\Enc \colon \pubkeys \times \adata \times \pubivs \times \secivs \times \ptxts \to \ctxts$ takes a public-key~$\pk\in\pubkeys$, associated data~$\header \in \adata$, public-IV~$\pubiv \in \pubivs$, secret-IV~$\seciv \in \secivs$ and a plaintext~$M \in \ptxts$, and returns a ciphertext~$C \in \ctxts$. 
When $\secivs$ is empty, then encryption is randomized.  Otherwise, it is \emph{IV-based} and deteministic.
To stress the differing semantics of the inputs, key and non-private/private data, we write $\Encprim{\pk}{\header,\pubiv}{S,M}$ instead of $\Enc(\pk,\header,\pubiv,\seciv,M)$ for the operation of encryption.  When encryption takes an oracle, we $\EncprimO{\pk}{\header,\pubiv}{\seciv,M}{\mathcal{O}}$ to clearly separate oracles from inputs.

The deterministic \emph{decryption} algorithm $\Dec \colon \seckeys \times \adata \times \pubivs \times \ctxts \to \left(\secivs \times \ptxts\right) \cup \{\bot\}$ takes a secret-key~$\sk\in\seckeys$, associated data~$\header \in \adata$, public-IV~$\pubiv \in \pubivs$, and a ciphertext~$C \in \ctxts$, and returns a pair $(S,M) \in \secivs\times\ptxts$, or the distinguished symbol~$\bot \not\in \ptxts$.  We write $(S,M) \gets \Decprim{\sk}{\header,\pubiv}{C}$ for the operation of decrpytion. 

For proper operation, we require that for all $(\pk,\sk)\in\pubkeys\times\seckeys$, $\header\in\adata$, $\pubiv\in\pubivs$, $\seciv\in\secivs$, and $M\in\ptxts$, we have $\Decprim{\sk}{\header,\pubiv}{\Encprim{\pk}{\header,\pubiv}{S,M}}=M$.


\paragraph{Symmetric AEAD Schemes. } \tsnote{Fill in similarly, may ultimately consolidate.}


\paragraph{Security notions. } 
Here we consider candidate security notions for an IV-based PK-AEAD scheme.  Fix a scheme $\pkaead=(\Kgen,\Enc,\Dec)$ and a randomized algorithm~$\advD$ that samples from the secret-IV set~$\secivs$ (of $\pkaead$) according to some distibution.  Let~$\mu_\advD$ be the min-entropy of this distribution.  Let~$\advA$ be an adversary.  

In Figure~\ref{fig:ind-cda} we give two notions of plaintext privacy.  In the first, the adversary queries an oracle that, on input $(\header,\pubiv,M_0,M_1)$, returns the encrpytion of $(\seciv,M_b)$ where~$S$ was previously sampled by~$\advD$, and~$b$ is the random challenge-bit.  In this notion, we assume that $|M_0|=|M_1|$ in every query and that~$\advA$ is \emph{nonce-respecting}, meaning that it never repeats a value of~$\pubiv$ across its queries.  We refer to this as the Priv security notion.
In the second notion, the adversary does not control the value of~$\pubiv$.  Instead, when~$\advA$ queries $(\header,M_0,M_1)$, a fresh public-IV~$\pubiv$ is sampled from the public-IV set~$\pubivs$.  The oracle returns~$\pubiv$ along with the encryption of $(S,M_b)$ using that particular public-IV.  We refer to this as Priv-IV security notion.

In practice, security with respect to the first notion means that the PK-AEAD scheme will protect plaintexts so long as the calling environment provides nonces for the IVs, e.g. a reliable counter value.  Security with respect to the second notion means that the PK-AEAD scheme guarantees privacy only when the calling environment provides IVs that are random and independent across calls.

\tsnote{There are other verions, too, that might make sense.  In one, $S\getsr\advD$ not once, but each time the oracle is queried.  There's no need for nonce respecting behavior then.  In another, the adversary can decide whether or not a new~$\seciv$ is sampled on a query, and it must be nonce-respecting for a given~$\seciv$.  An ``$\seciv$-adaptive version of this would allow~$\advA$ to refer to sampled values of~$\seciv$ by handles and arbitrarly interleave queries to handles.}
\begin{figure}
\begin{center}
\fpage{.5}{
 \hpagess{.425}{.55}{
 \underline{$\ExpINDCDA{\pkaead}{\advD,\advA}$}:\\[2pt]
 $(\pk,\sk)\getsr\Kgen$\\
 $b\getsr\bits$\\
 $S\getsr\advD$\\
 $b'\getsr\advA^{\encOracle(\cdot,\cdot,\cdot,\cdot)}(\pk)$\\
 Return $[b'=b]$\\

\medskip
 \underline{$\ExpINDCDAR{\pkaead}{\advD,\advA}$}:\\[2pt]
 $(\pk,\sk)\getsr\Kgen$\\
 $b\getsr\bits$\\
 $S\getsr\advD$\\
 $b'\getsr\advA^{\encOracle(\cdot,\cdot,\cdot)}(\pk)$\\
 Return $[b'=b]$

 }
 {
 \Oracle{$\OEnc(\header,\pubiv,M_0,M_1)$}:\\[2pt]
 Return $\Encprim{\pk}{\header,\pubiv}{S,M_b}$\\[51pt]

\medskip
 \Oracle{$\OEnc(\header,M_0,M_1)$}:\\[2pt]
 $N \getsr \pubivs$\\
 $C \gets  \Encprim{\pk}{\header,\pubiv}{S,M_b}$\\
 Return $(N,C)$

 }
}
\caption{Privacy notions for an IV-based PK-AEAD scheme~$\pkaead$ with public-IV and secret-IV sets $\pubivs,\secivs$ (resp.)}
\label{fig:ind-cda}
\end{center}
\end{figure}

For these two notions, we define corresponding advantage measures for a given adversary~$\advA$, sampler~$\advD$, and scheme $\pkaead$ as
\begin{align*}
\AdvINDCDA{\pkaead}{\advD,\advA}&=2\Prob{\ExpINDCDA{\pkaead}{\advD,\advA}=1}-1\\ \AdvINDCDAR{\pkaead}{\advD,\advA}&=2\Prob{\ExpINDCDAR{\pkaead}{\advD,\advA}=1}-1
\end{align*}
respectively.  In both experiments, we track the following adversarial resources: (1) the time-complexity~$t$ of~$\advA$, relative to some understood model of computation; (2) the query-complexity~$q$, measured as the number of queries made by~$\advA$ to its oracle; (3) the total query-length~$\sigma$, defined the be the sum over all query lengths, where $|(\header,\pubiv,M_0,M_1)|=|\header|+|\pubiv|+|M_0|$.


